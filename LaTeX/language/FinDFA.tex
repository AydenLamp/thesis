\chapter {Languages and Automata}

\section{DFA}

% Begin FinDFA/Defs.lean

\begin{definition}[Deterministic Finite Automaton (DFA)]
\label{def:DFA}
TODO: Description. Explain that mathlib has a definition `DFA` which does not necesarily 
have a finite state space or alphabet, and also does not require a decidable procedure for 
determining if a state is accepting or not. We created another definition `FinDFA` which does.
\lean{FinDFA, DFA} % Tag fully-qualified lean lemma names.
\leanok % use \leanok if the lean lemma above is complete
\end{definition}

\begin{definition}[Accessible DFA]
\label{def:AccessibleDFA}
TODO: Description. 
\lean{Accessible, AccessibleDFA}
\leanok 
\end{definition}

\begin{lemma}[TODO] \label{lem:MyLemma}
TODO: State that a state of a DFA is accessible iff it is accessible by some
word of length at most `|σ|`. 
\lean{Namespace.myLemma}
\leanok
\uses{def:MyDefinition, def:OtherDefinition} 
\end{lemma}

\begin{proof}
\leanok
Explain the proof in plain mathematical language.  Refer to earlier
results using `\ref{def:MyDefinition}` and list the proof dependencies
explicitly:
\uses{lem:OtherLemma, def:MyDefinition}
\end{proof}

\begin{definition}[Algorithm for deciding if a DFA state is Accessible]
\label{isAccessible}
We define a computable algorithm to decide if a state of a DFA is accessible. 
Refrence the previous lemma in justifying why it is computable. 
\lean{TODO}
\leanok 
\end{definition}

% End FinDFA/Defs.lean

TODO: Add a definition entry for the algorithm that produces an accessable DFA 
from any DFA accepting the same language (from Accessible.lean)

TODO: Definition entry for morphisms and Equivalences of DFAs.

TODO: Lemma entry - Morphisms preserve language

TODO: Lemma entry - THe relation defined by the existancee of surjective morphisms
on Accessable DFAs is a partial order up to equivalence.

TODO: Definition - Minimal dfa (by the previous preorder)

TODO: Definition - Distinguishing word

TODO: Definition - Nerode Eqivalence (also prove it is an equivalence relation)

TODO: Definition - Bounded nerode equivalence. Explain why this is computable

TODO : Lemma - Bounded nerode monotonicity

TODO: Lemma - Bounded nerode stabilization

TODO: Lemma - Nerode equivalence is the stabilization of bounded nerode equivalences.
Explain why this alows us to compute the Nerode equivalence.

TODO: Definition - Nerode automaton

TODO : theorem - THe nerode automaton is minimal and unique up to equivalence.





