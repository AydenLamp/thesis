% This file is included by both the web and print versions of the blueprint
% via an import statment in blueprint/scr/content.tex 

\section{Test}

Green's preorders are traditionally defined for semigroups by adjoining a unit element and working in the resulting monoid.  Here we start directly with monoids, thereby avoiding the overhead of the `WithOne` construction and the need to explicitly adjoin an identity.  Since a semigroup with an adjoined unit is essentially the same as a monoid, working on monoids simplifies many of the subsequent proofs.  In future developments this code might be refactored to operate on semigroups with a unit, but the monoid formulation suffices for our purposes.

\begin{definition}[Green's R-preorder]
\label{def:RRel}
Let \(M\) be a monoid and let \(x,y\in M\).  We define
\(x \le_R y\) if there exists \(z\in M\) such that \(x = y\cdot z\).
Equivalently, \(x\) lies in the principal right ideal generated by \(y\).
\end{definition}

\begin{lemma}[Reflexivity of \(\le_R\)]
\label{lem:RRel-refl}
For every \(x\in M\), \(x \le_R x\).
\uses{def:RRel}
\end{lemma}

\begin{lemma}[Transitivity of \(\le_R\)]
\label{lem:RRel-trans}
For all \(x,y,z\in M\), if \(x \le_R y\) and \(y \le_R z\) then \(x \le_R z\).
\uses{def:RRel}
\end{lemma}
\begin{proof}
Suppose \(x \le_R y\) and \(y \le_R z\).  By definition there exist \(v,u\in M\) with \(x = y\cdot v\) and \(y = z\cdot u\).  Taking the witness \(u\cdot v\) we compute \(z\cdot(u\cdot v) = (z\cdot u)\cdot v = y\cdot v = x\), so \(x \le_R z\).
\end{proof}