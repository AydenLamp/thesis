
\section{Inverses}

\noindent\textbf{Terminology note.}
The term “inverse” has two distinct usages. In group theory (and, more generally, in monoids), an inverse is defined using a distinguished identity element \(1\). This notion does not make sense in a bare semigroup that lacks a specified unit. Semigroup theory also uses a different, intrinsic notion of inverse that does not require a unit and is formulated purely in terms of the multiplication.

These notions behave differently:
\begin{itemize}
  \item In an \emph{infinite} monoid, an element may have several right group inverses and several left group inverses.
  \item In a \emph{finite} monoid, each element has \emph{at most one} right group inverse and \emph{at most one} left group inverse; if both exist, they coincide (hence give a two-sided group inverse).
  \item In a semigroup (finite or infinite), an element may have several semigroup inverses, or none at all.
\end{itemize}

\begin{definition}[Semigroup inverse]
\label{def:semigroup-inverse}
Let \(S\) be a semigroup and \(x\in S\). An element \(x'\in S\) is a \emph{semigroup inverse} of \(x\) if
\[
x x' x = x \quad\text{and}\quad x' x x' = x'.
\]
\end{definition}

\begin{definition}[Group inverse (monoid setting)]
\label{def:group-inverse}
Let \(M\) be a monoid with identity \(1\) and let \(x\in M\).
\begin{itemize}
  \item A \emph{right group inverse} of \(x\) is an element \(x'\in M\) with \(x x' = 1\).
  \item A \emph{left group inverse} of \(x\) is an element \(x'\in M\) with \(x' x = 1\).
  \item A \emph{group inverse} of \(x\) is an element \(x'\in M\) that is both a right and a left group inverse, i.e.\ \(x x' = x' x = 1\).
\end{itemize}
\end{definition}

\begin{lemma}[Group inverse \(\Rightarrow\) semigroup inverse]
\label{lem:group-inverse-implies-semigroup-inverse}
Let \(M\) be a monoid and \(x,x'\in M\). If \(x'\) is a group inverse of \(x\) (so \(x x' = x' x = 1\)), then \(x'\) is a semigroup inverse of \(x\) in the underlying semigroup:
\[
x x' x = x \quad\text{and}\quad x' x x' = x'.
\]
\uses{def:group-inverse,def:semigroup-inverse}
\end{lemma}
\begin{proof}
Compute \(x x' x = (x x') x = 1\cdot x = x\) and \(x' x x' = x' (x x') = x' \cdot 1 = x'\), using associativity and the unit laws.
\end{proof}