
\section{Zero elements and null semigroups}

Throughout, let \(S\) be a semigroup with associative multiplication, written multiplicatively.

\begin{definition}[Left/right/two-sided zeros]
\label{def:zeros}
Let \(e\in S\).
\begin{itemize}
  \item \(e\) is a \emph{left zero} if for all \(s\in S\), \(e s = e\).
  \item \(e\) is a \emph{right zero} if for all \(s\in S\), \(s e = e\).
  \item \(e\) is a \emph{zero} (two-sided) if it is both a left and a right zero; equivalently, for all \(s\in S\), \(e s = e = s e\).
\end{itemize}
\end{definition}

\begin{lemma}[Idempotence of one-sided zeros]
\label{lem:zero-idempotent}
Let \(e\in S\).
\begin{itemize}
  \item If \(e\) is a left zero, then \(e e = e\).
  \item If \(e\) is a right zero, then \(e e = e\).
\end{itemize}
\uses{def:zeros}
\end{lemma}
\begin{proof}
For a left zero, apply the defining property to \(s:=e\) to get \(e e = e\).
For a right zero, apply the defining property to \(s:=e\) to get \(e e = e\).
\end{proof}

\begin{lemma}[Uniqueness of zero (at most one zero element)]
\label{lem:zero-unique}
A semigroup has at most one zero element.
\uses{def:zero}
\end{lemma}
\begin{proof}
Suppose \(e,e'\in S\) are both zeros. Then \(e = e e'\) since \(e'\) is a right zero, and \(e e' = e'\) since \(e\) is a left zero. Hence \(e = e'\).
\end{proof}

\begin{definition}[Null semigroup]
\label{def:null-semigroup}
A semigroup \(S\) is \emph{null} if it has a zero element \(0_S\) and for all \(x,y\in S\) one has \(x y = 0_S\).
\uses{def:zero}
\end{definition}
