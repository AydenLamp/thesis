\chapter {Special Elements in Semigroups}

\section{Local identities in semigroups}

Throughout, let \(S\) be a semigroup with associative multiplication, written multiplicatively.

\begin{definition}[Left/right/two-sided identities]
\label{def:identities}
Let \(e\in S\).
\begin{itemize}
  \item \(e\) is a \emph{left identity} if for all \(s\in S\), \(e s = s\).
  \item \(e\) is a \emph{right identity} if for all \(s\in S\), \(s e = s\).
  \item \(e\) is an \emph{identity} (two-sided) if it is both a left and a right identity; equivalently, for all \(s\in S\), \(e s = s = s e\).
\end{itemize}
\end{definition}

\begin{lemma}[Idempotence of one-sided identities]
\label{lem:identity-idempotent}
Let \(e\in S\).
\begin{itemize}
  \item If \(e\) is a left identity, then \(e e = e\).
  \item If \(e\) is a right identity, then \(e e = e\).
\end{itemize}
\uses{def:identities}
\end{lemma}
\begin{proof}
For a left identity, apply the defining property to \(s:=e\) to get \(e e = e\).
For a right identity, apply the defining property to \(s:=e\) to get \(e e = e\).
\end{proof}

\begin{lemma}[Simplification lemma]
\label{lem:simplification}
Let \(s\in S\) and let \(e,f\in S\) be idempotents. If \(s = e s f\), then \(e s = s = s f\).
\end{lemma}
\begin{proof}
Assume \(s = e s f\). Then
\[
e s \;=\; e(e s f) \;=\; (e e) s f \;=\; e s f \;=\; s,
\]
using associativity and \(e^2 = e\). Similarly,
\[
s f \;=\; (e s f) f \;=\; e s (f f) \;=\; e s f \;=\; s,
\]
using associativity and \(f^2 = f\).
\end{proof}

\section{Zero elements and null semigroups}

Throughout, let \(S\) be a semigroup with associative multiplication, written multiplicatively.

\begin{definition}[Left/right/two-sided zeros]
\label{def:zeros}
Let \(e\in S\).
\begin{itemize}
  \item \(e\) is a \emph{left zero} if for all \(s\in S\), \(e s = e\).
  \item \(e\) is a \emph{right zero} if for all \(s\in S\), \(s e = e\).
  \item \(e\) is a \emph{zero} (two-sided) if it is both a left and a right zero; equivalently, for all \(s\in S\), \(e s = e = s e\).
\end{itemize}
\end{definition}

\begin{lemma}[Idempotence of one-sided zeros]
\label{lem:zero-idempotent}
Let \(e\in S\).
\begin{itemize}
  \item If \(e\) is a left zero, then \(e e = e\).
  \item If \(e\) is a right zero, then \(e e = e\).
\end{itemize}
\uses{def:zeros}
\end{lemma}
\begin{proof}
For a left zero, apply the defining property to \(s:=e\) to get \(e e = e\).
For a right zero, apply the defining property to \(s:=e\) to get \(e e = e\).
\end{proof}

\begin{lemma}[Uniqueness of zero (at most one zero element)]
\label{lem:zero-unique}
A semigroup has at most one zero element.
\uses{def:zero}
\end{lemma}
\begin{proof}
Suppose \(e,e'\in S\) are both zeros. Then \(e = e e'\) since \(e'\) is a right zero, and \(e e' = e'\) since \(e\) is a left zero. Hence \(e = e'\).
\end{proof}

\begin{definition}[Null semigroup]
\label{def:null-semigroup}
A semigroup \(S\) is \emph{null} if it has a zero element \(0_S\) and for all \(x,y\in S\) one has \(x y = 0_S\).
\uses{def:zero}
\end{definition}

\section{Cancellativity}

Throughout, let \(S\) be a semigroup with associative multiplication.

\begin{definition}[Right/left/two-sided cancellative element]
\label{def:cancellative-element}
Let \(s\in S\).
\begin{itemize}
  \item \(s\) is \emph{right cancellative} if for all \(x,y\in S\), \(x s = y s \implies x = y\).
  \item \(s\) is \emph{left cancellative} if for all \(x,y\in S\), \(s x = s y \implies x = y\).
  \item \(s\) is \emph{cancellative} (two-sided) if it is both left and right cancellative.
\end{itemize}
\end{definition}

\begin{definition}[Right/left/two-sided cancellative semigroup]
\label{def:cancellative-semigroup}
A semigroup \(S\) is
\begin{itemize}
  \item \emph{right cancellative} if every \(s\in S\) is right cancellative,
  \item \emph{left cancellative} if every \(s\in S\) is left cancellative,
  \item \emph{cancellative} (two-sided) if every \(s\in S\) is cancellative.
\end{itemize}
\uses{def:cancellative-element}
\end{definition}

\section{Inverses}

\noindent\textbf{Terminology note.}
The term “inverse” has two distinct usages. In group theory (and, more generally, in monoids), an inverse is defined using a distinguished identity element \(1\). This notion does not make sense in a bare semigroup that lacks a specified unit. Semigroup theory also uses a different, intrinsic notion of inverse that does not require a unit and is formulated purely in terms of the multiplication.

These notions behave differently:
\begin{itemize}
  \item In an \emph{infinite} monoid, an element may have several right group inverses and several left group inverses.
  \item In a \emph{finite} monoid, each element has \emph{at most one} right group inverse and \emph{at most one} left group inverse; if both exist, they coincide (hence give a two-sided group inverse).
  \item In a semigroup (finite or infinite), an element may have several semigroup inverses, or none at all.
\end{itemize}

\begin{definition}[Semigroup inverse]
\label{def:semigroup-inverse}
Let \(S\) be a semigroup and \(x\in S\). An element \(x'\in S\) is a \emph{semigroup inverse} of \(x\) if
\[
x x' x = x \quad\text{and}\quad x' x x' = x'.
\]
\end{definition}

\begin{definition}[Group inverse (monoid setting)]
\label{def:group-inverse}
Let \(M\) be a monoid with identity \(1\) and let \(x\in M\).
\begin{itemize}
  \item A \emph{right group inverse} of \(x\) is an element \(x'\in M\) with \(x x' = 1\).
  \item A \emph{left group inverse} of \(x\) is an element \(x'\in M\) with \(x' x = 1\).
  \item A \emph{group inverse} of \(x\) is an element \(x'\in M\) that is both a right and a left group inverse, i.e.\ \(x x' = x' x = 1\).
\end{itemize}
\end{definition}

\begin{lemma}[Group inverse \(\Rightarrow\) semigroup inverse]
\label{lem:group-inverse-implies-semigroup-inverse}
Let \(M\) be a monoid and \(x,x'\in M\). If \(x'\) is a group inverse of \(x\) (so \(x x' = x' x = 1\)), then \(x'\) is a semigroup inverse of \(x\) in the underlying semigroup:
\[
x x' x = x \quad\text{and}\quad x' x x' = x'.
\]
\uses{def:group-inverse,def:semigroup-inverse}
\end{lemma}
\begin{proof}
Compute \(x x' x = (x x') x = 1\cdot x = x\) and \(x' x x' = x' (x x') = x' \cdot 1 = x'\), using associativity and the unit laws.
\end{proof}

% --- NEW CHAPTER ---

\chapter {Ordered Semigroups and Monoids}

TODO: Def entry: Ordered Semigroup is a structure consisting of a semigroup together with a partial order that is compatable 
with the semigroup operation. Add a line of latex showing the implication of mul compatability.
In the same entry, define an ordered Monoid and ordered group analagously

% --- NEW CHAPTER ---

\chapter {Morphisms}

TODO: Def entry - Semigroup morphism.

 semigroup morphism is a map ϕ from a
semigroup S to a semigroup T such that, for every s1, s2 ∈ S, 
ϕ(s1s2) = ϕ(s1)ϕ(s2).

TODO: Def entry - Monoid Morphism
a monoid morphism is a map ϕ from a monoid S to a monoid T
satisfying (1) and
(2) ϕ(1) = 1

TODO: Def entry - Ordered Monoid Morphism
A morphism of ordered monoids is a map ϕ from an ordered monoid (S, 6) to
a monoid (T, 6) satisfying (1), (2) and, for every s1, s2 ∈ S such that s1 6 s2,
(3) ϕ(s1) 6 ϕ(s2).
Dependencies: Def of ordered monoid

TODO: Def entry - Group Morphism - Formally, a group morphism between two groups G and G′ is a monoid morphism
ϕ satisfying, for every s ∈ G, ϕ(s−1) = ϕ(s)−1. In fact, this condition can be
relaxed. 

TODO: Lemma entry 
Let G and G′ be groups. Then any semigroup morphism
from G to G′ is a group morphism.

Proof. Let ϕ : G → G′ be a semigroup morphism. Then by (1), ϕ(1) = ϕ(1)ϕ(1)
and thus ϕ(1) = 1 since 1 is the unique idempotent of G′. Thus ϕ is a monoid
morphism. Furthermore, ϕ(s−1)ϕ(s) = ϕ(s−1s) = ϕ(1) = 1 and similarly
ϕ(s)ϕ(s−1) = ϕ(ss−1) = ϕ(1) = 1.
Dependencies: Def of semigroup morphism, def of group morphism

TODO: Def entry - isomorphism
 A Semigroup morphism ϕ : S → T is an isomorphism if there exists a morphism ψ :
T → S such that ϕ ◦ ψ = IdT and ψ ◦ ϕ = IdS .
Note that this also holds for group morphisms and monoid morphisms, but not for ordered monoids
Dependencies: Defs of morphisms

TODO: Lemma entry - A Morphism is an isomorphism if and only if it is bijective.
Proof. If ϕ : S → T an isomorphism, then ϕ is bijective since there exists a
morphism ψ : T → S such that ϕ ◦ ψ = IdT and ψ ◦ ϕ = IdS .
Suppose now that ϕ : S → T is a bijective morphism. Then ϕ−1 is a
morphism from T to S, since, for each x, y ∈ T ,
ϕ(ϕ−1(x)ϕ−1(y)) = ϕ(ϕ−1(x))ϕ(ϕ−1(y)) = xy
Thus ϕ is an isomorphism.
Note that this does not hold for morphisms of ordered monoids. 
 In partic-
ular, if (M, 6) is an ordered monoid, the identity induces a bijective morphism
from (M, =) onto (M, 6) which is not in general an isomorphism
Dependencies: Defs of isomorphism

TODO: Def entry - Ordered monoid isomorphism
a morphism of ordered monoids ϕ : M → N is an isomorphism if and only if ϕ is
a bijective monoid morphism and, for every x, y ∈ M , x 6 y is equivalent with
ϕ(x) 6 ϕ(y).
Add Proof
Dependencies: Def of ordered monoid morphism 

\chapter {Algebraic Structures}

\section {Substructures}

TODO: Def entry - Subsemigroup 

A subsemigroup of a semigroup S is a subset T of S such that s1 ∈ T and s2 ∈ T
imply s1s2 ∈ T .

TODO: Def entry - SubMonoid of Monoid

A submonoid of a monoid is a subsemigroup containing the
identity

TODO: Def entry - Subgroup of Group

 A subgroup of a group is a submonoid containing the inverse of each
of its elements.

TODO: Def entry - SubMonoid of a Semigroup

A submonoid of a semigroup is a subsemigroup containing

A submonoid M of a semigroup S is said to be a monoid in S if there is
an idempotent e ∈ M such that M, under the operation of S, is a monoid with
identity e.

TODO: Def entry - Subgroup of Semigroup

A subsemigroup G of a semigroup S is said to be a group in S if there is
an idempotent e ∈ G such that G, under the operation of S, is a group with
identity e.

TODO: Lemma entry - 
This can be summarised as follows: substructures are preserved
by morphisms and by inverses of morphisms. A similar statement holds for
monoid morphisms and for group morphisms.
Let ϕ : S → T be a semigroup morphism. If S′ is a subsemi-
group of S, then ϕ(S′) is a subsemigroup of T . If T ′ is a subsemigroup of T ,
then ϕ−1(T ′) is a subsemigroup of S.
Proof. Let t1, t2 ∈ ϕ(S′). Then t1 = ϕ(s1) and t2 = ϕ(s2) for some s1, s2 ∈ S′.
Since S′ is a subsemigroup of S, s1s2 ∈ S′ and thus ϕ(s1s2) ∈ ϕ(S′). Now since
ϕ is a morphism, ϕ(s1s2) = ϕ(s1)ϕ(s2) = t1t2. Thus t1t2 ∈ ϕ(S′) and ϕ(S′) is
a subsemigroup of T .
Let s1, s2 ∈ ϕ−1(T ′). Then ϕ(s1), ϕ(s2) ∈ T ′ and since T ′ is a subsemigroup
of T , ϕ(s1)ϕ(s2) ∈ T ′. Since ϕ is a morphism, ϕ(s1)ϕ(s2) = ϕ(s1s2) and thus
s1s2 ∈ ϕ−1(T ′). Therefore ϕ−1(T ′) is a subsemigroup of S.
dependencies: Def of Morphisms, defs of substructures

TODO : Lemma entry: 
 A nonempty subsemigroup of a finite group is a subgroup.
Proof. Let G be a finite group and let S be a nonempty subsemigroup of G.
Let s ∈ S. By Proposition 3.12, s|G| = 1. Thus 1 ∈ S. Consider now the map
ϕ : S → S defined by ϕ(x) = xs. It is injective, for G is right cancellative, and
hence bijective by Proposition I.1.7. Consequently, there exists an element s′
such that s′s = 1. Thus every element has a left inverse and by Proposition 1.4,
S is a group.
Add the dependencies to this entry for the defs of subsemigroup and subgroups of groups

\section {Quotients and Divisions}

TODO: Before first entry, explain that the last two lemmas
prove that semigroup divison, when considered as a relation, forms 
a partial order on finite semigroups, up to isomorphism.

TODO : Def entry: Semigroup Quotient
Let S and T be two semigroups. Then T is
a quotient of S if there exists a surjective morphism from S onto T .
Dependencie: Def of semigroup morphism

TODO : Lemma entry: 
 any ordered monoid (M, 6) is a quotient of the ordered monoid
(M, =), since the identity on M is a morphism of ordered monoid from (M, =)
onto (M, 6).
Dependencie: Def of ordered monoid morphism

TODO : def entry: Semigroup Divisor

Finally, a semigroup T divides a semigroup S (notation T 4 S) if T is a
quotient of a subsemigroup of S.
Dependencie: Def of ordered quotient


TODO : Lemma entry : 
The division relation is transitive.
roof. Suppose that S1 4 S2 4 S3. Then there exists a subsemigroup T1
of S2, a subsemigroup T2 of S3 and surjective morphisms π1 : T1 → S1 and
π2 : T2 → S2. Put T = π−1
2 (T1). Then T is a subsemigroup of S3 and S1 is a
quotient of T since π1(π2(T )) = π1(T1) = S1. Thus S1 divides S3.
Add dempendencies for division relation and subsemigroup
Dependencie: Def of Division

TODO : Lemma entry :
the division relation is reflexive. That is, 
Two finite semigroups that divide each other are isomor-
phic.
Proof. We keep the notation of the proof of Proposition 3.9, with S3 = S1.
Since T1 is a subsemigroup of S2 and T2 is a subsemigroup of S1, one has |T1| 6
|S2| and |T2| 6 |S1|. Furthermore, since π1 and π2 are surjective, |S1| 6 |T1|
and |S2| 6 |T2|. It follows that |S1| = |T1| = |S2| = |T2|, whence T1 = S2
and T2 = S1. Furthermore, π1 and π2 are bijections and thus S1 and S2 are
isomorphic.
Dependencie: Def of Division

\chapter {Products}

TODO: Def entry: Semigroup Product
Given a list of semigroups, the product is the semigroup defined on the 
cartesian product of the sets, with the operation 
(si)i∈I (s′
i)i∈I = (sis′
i)i∈I

TODO : Lemma entry : The semigroup 1 is the identity for the product of semigroups [monoids,
groups]
TODO: Proof
Dependencie: Def of Product

TODO: Def - Product of ordered monoids
the product ∏
i∈I Mi is naturally
equipped with the order
(si)i∈I 6 (s′
i)i∈I if and only if, for all i ∈ I, si 6 s′
i.
The resulting ordered monoid is the product of the ordered monoids (Mi)i∈I .

TODO: Lemma entry : 
product preserves
substructures, quotients and division. We state it for semigroups, but it can be
readily extended to monoids and to ordered semigroups or monoids

Let (Si)i∈I and (Ti)i∈I be two families of semigroups such
that, for each i ∈ I, Si is a subsemigroup [quotient, divisor ] of Ti. Then ∏
i∈I Si
is a subsemigroup of [quotient, divisor ] of ∏
i∈I Ti.

Dependencie: Def of Product, Def of Division

\section {Ideals}

TODO: Def entry: Ideal (left, right, and two-sided)
Let S be a semigroup. A right ideal of S is a subset R of S such that RS ⊆ R.
Thus R is a right ideal if, for each r ∈ R and s ∈ S, rs ∈ R. Symmetrically, a
left ideal is a subset L of S such that SL ⊆ L. An ideal is a subset of S which
is simultaneously a right and a left ideal.

TODO: lemma: Semigroup ideal charactarization
For semigroup S, a subset I of S is an ideal if and only if, for every s ∈ I and
x, y ∈ S1, xsy ∈ I. Here, the use of S1 instead of S allows us to include the
cases x = 1 and y = 1, which are necessary to recover the conditions SI ⊆ S
and IS ⊆ I. Slight variations on the definition are therefore possible:
(1) R is a right ideal if and only if RS1 ⊆ R or, equivalently, RS1 = R,
(2) L is a left ideal if and only if S1L ⊆ L or, equivalently, S1L = L,
(3) I is an ideal if and only if S1IS1 ⊆ I or, equivalently, S1IS1 = I.
Dependencies : Def of ideal

TODO: lemma: Monoid ideal charactarization
For Monoid M a subset I of M is an ideal if and only if, for every m ∈ I and
x, y ∈ M, xmy ∈ I. Note that, in the case of monoids, we do not need the with one construction.
(1) R is a right ideal if and only if RM ⊆ R or, equivalently, RM = R,
(2) L is a left ideal if and only if ML ⊆ L or, equivalently, ML = L,
(3) I is an ideal if and only if MIM ⊆ I or, equivalently, MIM = I.
Dependencies : Def of ideal

TODO: Lemma
Note that any intersection of ideals [right ideals, left ideals] of S is again an
ideal [right ideal, left ideal].
TODO: proof
Dependencies : Def of ideal

TODO: Def entry: Ideal Generated by a set
Let R be a subset of a semigroup S. The ideal [right ideal, left ideal] gen-
erated by R is the set S1RS1 [RS1, S1R]. It is the smallest ideal [right ideal,
left ideal] containing R. 
Dependencies : Def of ideal

TODO: Def entry: Principal Ideals
An ideal [right ideal, left ideal] is called principal if
it is generated by a single element. 
Dependencies: Def of Ideal generated by a set.

TODO: lemma : Ideals are stable under surjective morphisms and inverses of morphisms. 
Let ϕ : S → T be a semigroup morphism. If J is an ideal
of T , then ϕ−1(J) is a ideal of S. Furthermore, if ϕ is surjective and I is an
ideal of S, then ϕ(I) is an ideal of T . Similar results apply to right and left
ideals.
Proof. If J is an ideal of T , then
S1ϕ−1(J)S1 ⊆ ϕ−1(T 1)ϕ−1(J)ϕ−1(T 1) ⊆ ϕ−1(T 1JT 1) ⊆ ϕ−1(J)
Thus ϕ−1(J) is an ideal of S.
Suppose that ϕ is surjective. If I is an ideal of S, then
T 1ϕ(I)T 1 = ϕ(S1)ϕ(I)ϕ(S1) = ϕ(S1IS1) = ϕ(I)
Thus ϕ(I) is an ideal of T .

Dependencies: Def of Ideal, Def of morphism

TODO: Def: Product of ideals : 
Let, for 1 6 k 6 n, Ik be an ideal of a semigroup S. The set
I1I2 · · · In = {s1s2 · · · sn | s1 ∈ I1, s2 ∈ I2, . . . , sn ∈ In}
is the product of the ideals I1, . . . , In.
Dependencies: Def of ideals

TODO : Lemma : The product of the ideals I1, . . . , In is an ideal contained in
their intersection.
Proof. Since I1 and In are ideals, S1I1 = I1 and InS1 = In. Therefore
S1(I1I2 · · · In)S1 = (S1I1)I2 · · · (InS1) = I1I2 · · · In
and thus I1I2 · · · In is an ideal. Furthermore, for 1 6 k 6 n, I1I2 · · · In ⊆
S1IkS1 = Ik. Thus I1I2 · · · In is contained in ⋂
16k6n Ik.
Dependencies: Def of product of ideals. 

TODO: Def : Minimal Ideal
A nonempty ideal I of a semigroup S is called minimal if, for every nonempty
ideal J of S, J ⊆ I implies J = I.
Dependencies: def of ideal

TODO : Lemma :  semigroup has at most one minimal ideal.
Proof. Let I1 and I2 be two minimal ideals of a semigroup S. Then by(link earlier lemma), I1I2 is a nonempty ideal of S contained in I1 ∩ I2. Now since I1 and
I2 are minimal ideals, I1I2 = I1 = I2
Dependencies: Def of minimal ideas, Lemma product of ideals in contained in intersection

TODO: Lemma : Any finite semigroup has a minimal ideal.
TODO: Proof
Dependencies: def of minimal ideal

ToDO: lemma: every semigroup with a zero has a minimal ideal, namely the set containing only the zero.
TODO: Proof
Dependencies: def of minimal ideal, def of zero

TODO: Def: 0-minimal ideal
. A
nonempty ideal I 6 = 0 such that, for every nonempty ideal J of S, J ⊆ I implies
J = 0 or J = I is called a 0-minimal ideal. It should be noted that a semigroup
may have several 0-minimal ideals.
Dependencies: def of ideal, def of zero

\section {Simple and 0-Ssimple semigroups}

TODO: Def entry: Simple Semigroup
A semigroup S is called simple if its only ideals are ∅ and S.
right and left simple are defined analogously
Dependencies: Def of ideal

TODO: Def entry: 0-Simple Semigroup
 0-simple if it has a zero, denoted by 0, if S2 6 = {0} and if ∅, 0 and S are its only
ideals. The notions of right 0-simple, and left 0-simple
semigroups are defined analogously.
Dependencies: Def of ideal, def of zero

TODO : Lemma : et S be a 0-simple semigroup. Then S2 = S.
Proof. Since S2 is a nonempty, nonzero ideal, one has S2 = S.
DEPendencies: def of 0-simple semigroup

TODO: Lemma: 
(1) A semigroup S is simple if and only if SsS = S for every s ∈ S.
(2) A semigroup S is 0-simple if and only if S 6 = ∅ and SsS = S for every
s ∈ S − 0.
Proof. We shall prove only (2), but the proof of (1) is similar.
Let S be a 0-simple semigroup. Then S2 = S by Lemma 3.18 and hence
S3 = S.
Let I be set of the elements s of S such that SsS = 0. This set is an ideal of
S containing 0, but not equal to S since ⋃
s∈S SsS = S3 = S. Therefore I = 0.
In particular, if s 6 = 0, then SsS 6 = 0, and since SsS is an ideal of S, it follows
that SsS = S.
Conversely, if S 6 = ∅ and SsS = S for every s ∈ S −0, we have S = SsS ⊆ S2
and therefore S2 6 = 0. Moreover, if J is a nonzero ideal of S, it contains an
element s 6 = 0. We then have S = SsS ⊆ SJS = J, whence S = J. Therefore
S is 0-simple.
DEPendencies: def of 0-simple semigroup, def of Simple semigroup

\section {Semigroup Congruences}

TODO: Def entry: Semigroup Congruence
A semigroup congruence is a stable equivalence relation. Thus an equivalence
relation ∼ on a semigroup S is a congruence if, for each s, t ∈ S and u, v ∈ S1,
we have
s ∼ t implies usv ∼ utv.

TODO: Lemma : The set S/∼ of equivalence classes of the elements of S is naturally equipped
with the structure of a semigroup, and the function which maps every element
onto its equivalence class is a semigroup morphism from S onto S/∼
Dependences: Def of congruence, def of quotient

There are four particular cases of congruences extensivly used, Rees congruence, Syntactic congruence, Congruence 
generated by a relation, and Nuclear Congruence

TODO: Def : Rees Congruence
et I be an ideal of a semigroup S and let ≡I be the equivalence relation
identifying all the elements of I and separating the other elements. Formally,
s ≡I t if and only if s = t or s, t ∈ I. Then ≡I is a congruence called the
Rees congruence of I. The quotient of S by ≡I is usually denoted by S/I. The
support of this semigroup is the set (S − I) ∪ 0 and the multiplication (here
denoted by ∗) is defined by
s ∗ t =
{
st if s, t, st ∈ S − I
0 otherwise.
Dependences: Def of congruence, def of quotient, def of ideal

TODO: Def: Syntactic congruence
Let P be a subset of a semigroup S. The syntactic congruence of P is the
congruence ∼P over S defined by s ∼P t if and only if, for every x, y ∈ S1,
xsy ∈ P ⇐⇒ xty ∈ P
The quotient semigroup S/∼P is called the syntactic semigroup of P in S. The
syntactic semigroup is particularly important in the theory of formal languages.
Dependences: Def of congruence, def of quotient

TODO: Def: Congruence generated by a relation
Let R be a relation on S, that is, a subset of S × S. The set of all congruences
containing R is nonempty since it contains the universal relation on S. Fur-
thermore, it is closed under intersection. It follows that the intersection of all
congruences containing R is a congruence, called the congruence generated by
R. 
Dependences: Def of congruence

TODO: Lemma: he congruence generated by a symmetric relation R on a
semigroup S is the reflexive-transitive closure of the relation
{(xry, xsy) | (r, s) ∈ R and x, y ∈ S1}
Proof. If a congruence contains R, it certainly contains the relation
¯R = {(xry, xsy) | (r, s) ∈ R and x, y ∈ S1}
and hence its reflexive-transitive closure ¯R∗. Therefore, it suffices to show that
¯R∗ is a congruence. Let (u, v) ∈ ¯R∗. By definition, there exists a finite sequence
u = u0, u1, . . . , un = v such that
(u0, u1) ∈ ¯R, (u1, u2) ∈ ¯R, · · · , (un−1, un) ∈ ¯R
Therefore, one has, for some xi, yi, ri, si ∈ S such that (ri, si) ∈ R,
(u0, u1) = (x0r0y0, x0s0y0), (u1, u2) = (x1r1y1, x1s1y1), · · · ,
(un−1, un) = (xn−1rn−1yn−1, xn−1sn−1yn−1)
Let now x, y ∈ S1. Then the relations
(xuiy, xui+1y) = (xxiriyiy, xxisiyiy) ∈ ¯R (0 6 i 6 n − 1)
show that (xuy, xvy) ∈ ¯R∗. Thus ¯R∗ is a congruence.
Dependencies: Def of congruence generated by a relation

TODO: Def: Nuclear Congruence
Nuclear congruence
For each semigroup morphism ϕ : S → T , the equivalence ∼ϕ defined on S by
x ∼ϕ y if and only if ϕ(x) = ϕ(y)
is a congruence. This congruence is called the nuclear congruence of ϕ,
DEPENCENCIES: def of morphism, Def of congruence

TODO: Lemma : (First isomorphism theorem). Let ϕ : S → T be a morphism
of semigroups and let π : S → S/∼ϕ be the quotient morphism. Then there exists
a unique semigroup morphism ˜ϕ : S/∼ϕ → T such that ϕ = ˜ϕ ◦ π. Moreover,
˜ϕ is an isomorphism from S/∼ϕ onto ϕ(S).
26 CHAPTER II. SEMIGROUPS AND BEYOND
Proof. The situation is summed up in the following diagram:
S/∼ϕ T
S
π
˜ϕ
ϕ
Uniqueness is clear: if s is the ∼ϕ-class of some element x, then necessarily
˜ϕ(s) = ϕ(x) (3.1)
Furthermore, if x and y are arbitrary elements of s, then ϕ(x) = ϕ(y). Therefore,
there is a well-defined function ˜ϕ defined by the Formula. Moreover, if π(x1) =
s1 and π(x2) = s2, then π(x1x2) = s1s2, whence
˜ϕ(s1) ˜ϕ(s2) = ϕ(x1)ϕ(x2) = ϕ(x1x2) = ˜ϕ(s1s2)
Therefore ˜ϕ is a morphism. We claim that ˜ϕ is injective. Indeed, suppose that
˜ϕ(s1) = ˜ϕ(s2), and let x1 ∈ π−1(s1) and x2 ∈ π−1(s2). Then ϕ(x1) = ϕ(x2)
and thus x1 ∼ϕ x2. It follows that π(x1) = π(x2), that is, s1 = s2. Thus ˜ϕ
induces an isomorphism from S/∼ϕ onto ϕ(S)
Dependencies: Def of nuclear congruence, def of quotient, def of isomorphism

When two congruences are comparable, the quotient structures associated
with them can also be compared

TODO: Theorem : (Second isomorphism theorem). Let ∼1 and ∼2 be two con-
gruences on a semigroup S and π1 [π2] the canonical morphism from S onto
S/∼1 [S/∼2]. If ∼2 is coarser than ∼1, there exists a unique surjective mor-
phism π : S/∼1 → S/∼2 such that π ◦ π1 = π2.
Proof. Since π ◦ π1 = π2, Corollary I.1.13 shows that π is necessarily equal to
the relation π2 ◦ π−1
1 . Furthermore, Proposition I.1.15 shows that this relation
is actually a function.
Since π1 and π2 are morphisms,
π(π1(s)π1(t)) = π(π1(st)) = π2(st) = π2(s)π2(t) = π(π1(s))π(π1(t))
and thus π is a morphism.
Dependencies: Def of congruence, def of quotient

TODO: Lemma
Let S be a semigroup, (∼i)i∈I be a family of congruences
on S and ∼ be the intersection of these congruences. Then the semigroup S/∼
is isomorphic to a subsemigroup of the product ∏
i∈I S/∼i.
Proof. Denote by πi : S → S/∼i the projections and by π : S → ∏
i∈I S/∼i the
morphism defined by π(s) = (πi(s))i∈I for every s ∈ S. The nuclear congruence
of π is equal to ∼, and thus, by Proposition 3.21, S/∼ is isomorphic to π(S)
Dependencies: Def of congruence, def of quotient, doef of isomorphism, def of subsemigroup, def of semigroup product. 

TODO: lemma
Let S be a semigroup with zero having (at least) two distinct
0-minimal ideals I1 and I2. Then S is isomorphic to a subsemigroup of S/I1 ×
S/I2.
4. TRANSFORMATION SEMIGROUPS 27
Proof. Under these assumptions, the intersection of I1 and I2 is 0 and thus the
intersection of the Rees congruences ≡I1 and ≡I2 is the identity. It remains to
apply the previous lemma.
Dependencies: Previous lemma, 0-minimal ideals

tODO: DEF:Let (M, 6) be an ordered monoid. A congruence of ordered semigroups
on M is a stable preorder coarser than 6. If  is a congruence of ordered
monoids, the equivalence ∼ associated with  is a semigroup congruence and
the preorder  induces a stable order relation on the quotient monoid S/∼, that
will also be denoted by 6. Finally, the function which maps each element onto
its equivalence class is a morphism of ordered monoids from M onto (M/∼, 6).
Dependencies: Def of ordered monoid, def of congruence, def of quotient
