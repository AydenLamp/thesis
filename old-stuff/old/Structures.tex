
%%%%%%%%%%%%%%%%%%%%%%%%%%%%%%%%%%%%%%%%%%%%%%%%%%%%%%%%%%%%%%%%%%%

\section{Substructures}

\begin{definition}[Subsemigroup]
\label{def:subsemigroup}
A \emph{subsemigroup} of a semigroup \(S\) is a nonempty subset \(T\subseteq S\) such that for all \(t_1,t_2\in T\), one has \(t_1t_2\in T\).
\end{definition}

\begin{definition}[Submonoid of a monoid]
\label{def:submonoid}
A \emph{submonoid} of a monoid \(M\) is a subsemigroup \(T\subseteq M\) containing the identity \(1_M\).
\uses{def:subsemigroup}
\end{definition}

\begin{definition}[Subgroup of a group]
\label{def:subgroup}
A \emph{subgroup} of a group \(G\) is a submonoid \(H\subseteq G\) that is closed under inversion: \(h\in H\Rightarrow h^{-1}\in H\).
\uses{def:submonoid}
\end{definition}

\begin{definition}[Monoid/group inside a semigroup]
\label{def:internal-monoid-group}
Let \(S\) be a semigroup.
\begin{itemize}
  \item A subsemigroup \(M\subseteq S\) is a \emph{monoid in \(S\)} if there exists an idempotent \(e\in M\) such that \((M,\cdot,e)\) is a monoid (with identity \(e\)) under the inherited multiplication.
  \item A subsemigroup \(G\subseteq S\) is a \emph{group in \(S\)} if there exists an idempotent \(e\in G\) such that \((G,\cdot,e,(\cdot)^{-1})\) is a group under the inherited multiplication.
\end{itemize}
\uses{lem:identity-idempotent}
\end{definition}

\begin{lemma}[Images and preimages preserve substructures]
\label{lem:morphism-preserves-substructures}
Let \(\varphi:S\to T\) be a semigroup morphism.
\begin{itemize}
  \item If \(S'\subseteq S\) is a subsemigroup, then \(\varphi(S')\) is a subsemigroup of \(T\).
  \item If \(T'\subseteq T\) is a subsemigroup, then \(\varphi^{-1}(T')\) is a subsemigroup of \(S\).
\end{itemize}
Analogous statements hold for monoid and group morphisms and their corresponding substructures.
\uses{def:semigroup-morphism,def:subsemigroup}
\end{lemma}
\begin{proof}
If \(t_1=\varphi(s_1)\in \varphi(S')\) and \(t_2=\varphi(s_2)\in \varphi(S')\) with \(s_1,s_2\in S'\), then \(t_1t_2=\varphi(s_1)\varphi(s_2)=\varphi(s_1s_2)\in\varphi(S')\). For preimages: if \(s_1,s_2\in \varphi^{-1}(T')\), then \(\varphi(s_i)\in T'\) and \(\varphi(s_1s_2)=\varphi(s_1)\varphi(s_2)\in T'\), hence \(s_1s_2\in \varphi^{-1}(T')\).
\end{proof}

\begin{lemma}[Finite-group test]
\label{lem:finite-group-subsemigroup-is-subgroup}
A nonempty subsemigroup \(S'\) of a finite group \(G\) is a subgroup of \(G\).
\uses{def:subsemigroup,def:subgroup}
\end{lemma}
\begin{proof}
Pick \(g\in S'\). The set \(\{g^n\mid n\ge 1\}\subseteq S'\) is finite, so \(g^i=g^j\) with \(1\le i<j\). By cancellation in \(G\), \(g^{j-i}=1\in S'\). For any \(h\in S'\), the set \(\{h^n\mid n\ge 0\}\) is finite, hence \(h^a=h^b\) with \(a<b\). Cancelling \(h^a\) yields \(1=h^{b-a}\in S'\); then \(h^{b-a-1}\in S'\) is an inverse of \(h\). Thus \(S'\) is a subgroup.
\end{proof}

%%%%%%%%%%%%%%%%%%%%%%%%%%%%%%%%%%%%%%%%%%%%%%%%%%%%%%%%%%%%%%%%%%%

\section {Quotients and Divisions}

\noindent
In this section, “quotient" means “image of a surjective morphism”. For finite semigroups, the last two lemmas below show that the \emph{division} relation (defined via quotients of subsemigroups) is a partial order on isomorphism classes.

\begin{definition}[Semigroup quotient]
\label{def:semigroup-quotient}
A semigroup \(T\) is a \emph{quotient} of a semigroup \(S\) if there exists a surjective semigroup morphism \(\pi:S\twoheadrightarrow T\).
\uses{def:semigroup-morphism}
\end{definition}

\begin{lemma}[Trivial order arises as a quotient]
\label{lem:ordered-monoid-is-quotient-of-equality}
For any ordered monoid \((M,\le)\), the identity map \(\mathrm{id}_M:(M,=)\to (M,\le)\) is a surjective morphism of ordered monoids. Hence \((M,\le)\) is a quotient of \((M,=)\) in the ordered-monoid sense.
\uses{def:ordered-monoid-morphism}
\end{lemma}
\begin{proof}
The identity preserves the monoid structure and is order-preserving from equality to \(\le\) trivially.
\end{proof}

\begin{definition}[Division (divisor)]
\label{def:division}
A semigroup \(T\) \emph{divides} a semigroup \(S\) (notation \(T \mid S\)) if there exist a subsemigroup \(U\subseteq S\) and a surjective morphism \(\pi:U\twoheadrightarrow T\). Equivalently, \(T\) is a quotient of a subsemigroup of \(S\).
\uses{def:subsemigroup,def:semigroup-quotient}
\end{definition}

\begin{lemma}[Transitivity of division]
\label{lem:division-transitive}
If \(S_1\mid S_2\) and \(S_2\mid S_3\), then \(S_1\mid S_3\).
\uses{def:division}
\end{lemma}
\begin{proof}
Let \(U_1\subseteq S_2\) and \(\pi_1:U_1\twoheadrightarrow S_1\) witness \(S_1\mid S_2\), and \(U_2\subseteq S_3\) and \(\pi_2:U_2\twoheadrightarrow S_2\) witness \(S_2\mid S_3\). Then \(U:=\pi_2^{-1}(U_1)\subseteq S_3\) is a subsemigroup and \(\pi_1\circ \pi_2:U\twoheadrightarrow S_1\) is surjective, so \(S_1\mid S_3\).
\end{proof}

\begin{lemma}[Reflexivity and antisymmetry on finite semigroups]
\label{lem:division-reflexive-antisymmetric-finite}
\leavevmode
\begin{enumerate}
\item For any semigroup \(S\), \(S\mid S\) (take \(U=S\) and \(\pi=\mathrm{id}\)).
\item If \(S_1,S_2\) are finite and \(S_1\mid S_2\) and \(S_2\mid S_1\), then \(S_1\cong S_2\).
\end{enumerate}
\uses{def:division,def:isomorphism}
\end{lemma}
\begin{proof}
(1) is immediate. For (2), choose \(U_1\subseteq S_2\) with \(\pi_1:U_1\twoheadrightarrow S_1\) and \(U_2\subseteq S_1\) with \(\pi_2:U_2\twoheadrightarrow S_2\). Then
\(|S_1|\le |U_1|\le |S_2|\) and \(|S_2|\le |U_2|\le |S_1|\), hence all are equalities:
\(|S_1|=|U_1|=|S_2|=|U_2|\). Therefore \(U_1=S_2\), \(U_2=S_1\) and both \(\pi_1,\pi_2\) are bijections, i.e.\ isomorphisms. Thus \(S_1\cong S_2\).
\end{proof}

%%%%%%%%%%%%%%%%%%%%%%%%%%%%%%%%%%%%%%%%%%%%%%%%%%%%%%%%%%%%%%%%%%%

\section {Products}

\begin{definition}[Product of semigroups]
\label{def:semigroup-product}
Given a family \(\{S_i\}_{i\in I}\) of semigroups, the Cartesian product \(\prod_{i\in I} S_i\) is a semigroup under coordinatewise multiplication:
\[
(s_i)_{i\in I}\cdot (t_i)_{i\in I} \;:=\; (\,s_i t_i\,)_{i\in I}.
\]
\end{definition}

\begin{lemma}[Unit object for products]
\label{lem:one-is-identity-for-product}
Let \(1\) denote the one-element semigroup (resp.\ monoid, group). Then \(S\times 1\cong S\cong 1\times S\) via the coordinate projections; these are semigroup (resp.\ monoid, group) isomorphisms.
\uses{def:semigroup-product,def:isomorphism}
\end{lemma}
\begin{proof}
The maps \((s,\ast)\mapsto s\) and \(s\mapsto (s,\ast)\) are mutually inverse morphisms.
\end{proof}

\begin{definition}[Product of ordered monoids]
\label{def:ordered-monoid-product}
If \(\{(M_i,\le_i)\}_{i\in I}\) are ordered monoids, their product ordered monoid is \(\big(\prod_i M_i,\le\big)\) where
\[
(s_i)_{i\in I}\le (t_i)_{i\in I}\quad\Longleftrightarrow\quad \forall i\in I,\; s_i\le_i t_i.
\]
This order is compatible with the coordinatewise multiplication.
\uses{def:ordered-structures}
\end{definition}

\begin{lemma}[Products preserve substructures, quotients, and division]
\label{lem:products-preserve-structure}
Let \(\{S_i\}_{i\in I}\) and \(\{T_i\}_{i\in I}\) be families of semigroups.
\begin{itemize}
  \item If each \(S_i\subseteq T_i\) is a subsemigroup, then \(\prod_i S_i\subseteq \prod_i T_i\) is a subsemigroup.
  \item If each \(\pi_i:T_i\twoheadrightarrow S_i\) is a surjective morphism, then \(\prod_i \pi_i:\prod_i T_i\twoheadrightarrow \prod_i S_i\) is a surjective morphism; hence products of quotients are quotients of products.
  \item If each \(S_i\) divides \(T_i\), then \(\prod_i S_i\) divides \(\prod_i T_i\).
\end{itemize}
\uses{def:subsemigroup,def:semigroup-quotient,def:division,def:semigroup-product}
\end{lemma}
\begin{proof}
All three items are checked coordinatewise.
\end{proof}

%%%%%%%%%%%%%%%%%%%%%%%%%%%%%%%%%%%%%%%%%%%%%%%%%%%%%%%%%%%%%%%%%%%

\section {Ideals}

For a semigroup \(S\), write \(S^1\) for \(S\) if \(S\) already has an identity, and otherwise for the semigroup obtained by adjoining a new identity \(1\). This allows uniform formulations using \(S^1\).

\begin{definition}[Right/left/two-sided ideals]
\label{def:ideal}
Let \(S\) be a semigroup. A \emph{right ideal} is a subset \(R\subseteq S\) with \(RS\subseteq R\). A \emph{left ideal} is a subset \(L\subseteq S\) with \(SL\subseteq L\). An \emph{ideal} is a subset \(I\subseteq S\) with \(SI\subseteq I\) and \(IS\subseteq I\).
\end{definition}

\begin{lemma}[Characterizations via \(S^1\)]
\label{lem:ideal-characterization-semigroup}
Let \(S\) be a semigroup and \(I\subseteq S\).
\begin{enumerate}
  \item \(I\) is a right ideal iff \(I S^1=I\) (equivalently, \(I S^1\subseteq I\)).
  \item \(I\) is a left ideal iff \(S^1 I=I\) (equivalently, \(S^1 I\subseteq I\)).
  \item \(I\) is an ideal iff \(S^1 I S^1=I\) (equivalently, \(S^1 I S^1\subseteq I\)).
\end{enumerate}
\uses{def:ideal,rem:S1}
\end{lemma}
\begin{proof}
If \(I\) is a right ideal then \(IS\subseteq I\); multiplying by \(1\in S^1\) gives \(IS^1\subseteq I\) and \(\supseteq\) is clear. The other items are analogous; for (3) combine the previous two.
\end{proof}

\begin{lemma}[Monoid case]
\label{lem:ideal-characterization-monoid}
If \(M\) is a monoid and \(I\subseteq M\), then \(I\) is a right ideal iff \(IM=I\), a left ideal iff \(MI=I\), and an ideal iff \(MIM=I\).
\uses{def:ideal}
\end{lemma}
\begin{proof}
Same as above with \(S^1=M\).
\end{proof}

\begin{lemma}[Intersections]
\label{lem:intersection-of-ideals}
Arbitrary intersections of (right/left/two-sided) ideals are (right/left/two-sided) ideals.
\uses{def:ideal}
\end{lemma}
\begin{proof}
If \(\{I_\alpha\}\) are ideals, then \(S(\bigcap_\alpha I_\alpha)\subseteq \bigcap_\alpha S I_\alpha\subseteq \bigcap_\alpha I_\alpha\), and similarly on the right.
\end{proof}

\begin{definition}[Ideal generated by a set; principal ideals]
\label{def:generated-ideal}
For \(R\subseteq S\), the ideal generated by \(R\) is \(S^1 R S^1\); the right (resp.\ left) ideal generated by \(R\) is \(R S^1\) (resp.\ \(S^1 R\)). An ideal is \emph{principal} if it is generated by a single element.
\uses{def:ideal,rem:S1}
\end{definition}

\begin{lemma}[Ideals and morphisms]
\label{lem:ideals-stable-under-morphisms}
Let \(\varphi:S\to T\) be a semigroup morphism. If \(J\subseteq T\) is an ideal, then \(\varphi^{-1}(J)\) is an ideal of \(S\). If \(\varphi\) is surjective and \(I\subseteq S\) is an ideal, then \(\varphi(I)\) is an ideal of \(T\).
\uses{def:ideal,def:semigroup-morphism}
\end{lemma}
\begin{proof}
For preimages: \(S^1\varphi^{-1}(J)S^1\subseteq \varphi^{-1}(T^1)\varphi^{-1}(J)\varphi^{-1}(T^1)\subseteq \varphi^{-1}(T^1JT^1)=\varphi^{-1}(J)\). For images with \(\varphi\) surjective: \(T^1\varphi(I)T^1=\varphi(S^1)\varphi(I)\varphi(S^1)=\varphi(S^1 I S^1)=\varphi(I)\).
\end{proof}

\begin{definition}[Product of ideals]
\label{def:product-of-ideals}
If \(I_1,\dots,I_n\) are ideals of \(S\), their \emph{product} is
\[
I_1\cdots I_n:=\{\,s_1\cdots s_n \mid s_k\in I_k\,\}.
\]
\uses{def:ideal}
\end{definition}

\begin{lemma}[Product of ideals]
\label{lem:product-ideal-in-intersection}
The product \(I_1\cdots I_n\) is an ideal and \(I_1\cdots I_n\subseteq \bigcap_{k=1}^n I_k\).
\uses{def:product-of-ideals,def:ideal,rem:S1}
\end{lemma}
\begin{proof}
Since \(S^1 I_1=I_1\) and \(I_n S^1=I_n\), we have \(S^1(I_1\cdots I_n)S^1=(S^1 I_1)\cdots (I_n S^1)=I_1\cdots I_n\). For inclusion, fix \(k\); then \(I_1\cdots I_n\subseteq S^1 I_k S^1=I_k\).
\end{proof}

\begin{definition}[Minimal and \(0\)-minimal ideals]
\label{def:minimal-ideal}
A nonempty ideal \(I\) of \(S\) is \emph{minimal} if \(J\subseteq I\) and \(J\) an ideal implies \(J=I\). If \(S\) has a zero \(0\), a nonempty ideal \(I\ne \{0\}\) is \emph{\(0\)-minimal} if every nonempty ideal \(J\subseteq I\) satisfies \(J=\{0\}\) or \(J=I\).
\uses{def:ideal,def:zeros}
\end{definition}

\begin{lemma}[Uniqueness of a minimal ideal]
\label{lem:at-most-one-minimal-ideal}
A semigroup has at most one minimal ideal.
\uses{def:minimal-ideal,lem:product-ideal-in-intersection}
\end{lemma}
\begin{proof}
If \(I_1,I_2\) are minimal ideals, then \(I_1 I_2\) is a nonempty ideal contained in \(I_1\cap I_2\). By minimality, \(I_1 I_2=I_1=I_2\).
\end{proof}

\begin{lemma}[Existence in the finite case]
\label{lem:finite-semigroup-has-minimal-ideal}
Every finite semigroup has a minimal ideal.
\uses{def:minimal-ideal}
\end{lemma}
\begin{proof}
Among the nonempty ideals (nonempty because singletons \(\{s\}\) generate ideals), pick one of minimal cardinality; it is minimal by definition.
\end{proof}

\begin{lemma}[Zero yields a minimal ideal]
\label{lem:zero-gives-minimal-ideal}
If \(S\) has a zero \(0\), then \(\{0\}\) is a minimal ideal.
\uses{def:minimal-ideal,def:zeros}
\end{lemma}
\begin{proof}
\(\{0\}\) is an ideal since \(S\{0\}=\{0\}=\{0\}S\). If \(J\subseteq \{0\}\) is a nonempty ideal, then \(J=\{0\}\).
\end{proof}

%%%%%%%%%%%%%%%%%%%%%%%%%%%%%%%%%%%%%%%%%%%%%%%%%%%%%%%%%%%%%%%%%%%

\section {Simple and \(0\)-Simple semigroups}

\begin{definition}[Simple and \(0\)-simple]
\label{def:simple-zero-simple}
A semigroup \(S\) is \emph{simple} if its only ideals are \(\varnothing\) and \(S\). If \(S\) has a zero \(0\), then \(S\) is \emph{\(0\)-simple} if \(S^2\ne \{0\}\) and the only ideals are \(\varnothing,\{0\},S\). One-sided versions (right/left simple, right/left \(0\)-simple) are defined analogously.
\uses{def:ideal,def:zeros}
\end{definition}

\begin{lemma}
\label{lem:zero-simple-square}
If \(S\) is \(0\)-simple, then \(S^2=S\).
\uses{def:simple-zero-simple}
\end{lemma}
\begin{proof}
\(S^2\) is a nonempty, nonzero ideal, hence \(S^2=S\).
\end{proof}

\begin{lemma}[Characterizations via principal two-sided ideals]
\label{lem:simple-characterization}
\leavevmode
\begin{enumerate}
  \item \(S\) is simple iff \(S s S=S\) for every \(s\in S\).
  \item If \(S\ne\varnothing\) and has a zero \(0\), then \(S\) is \(0\)-simple iff \(S s S=S\) for every \(s\in S\setminus\{0\}\).
\end{enumerate}
\uses{def:simple-zero-simple,lem:zero-simple-square,def:generated-ideal}
\end{lemma}
\begin{proof}
(2) Suppose \(S\) is \(0\)-simple. By Lemma~\ref{lem:zero-simple-square}, \(S^2=S\), so \(\bigcup_{s\in S} SsS=S\). The set \(I=\{s\in S\mid SsS=\{0\}\}\) is an ideal containing \(0\) but not equal to \(S\); hence \(I=\{0\}\). Thus for \(s\ne 0\), \(SsS\ne \{0\}\), and being an ideal, it equals \(S\). Conversely, if \(S\ne\varnothing\) and \(SsS=S\) for all \(s\ne 0\), then \(S^2\ne \{0\}\) and any nonzero ideal \(J\) contains some \(s\ne 0\), hence \(S=SsS\subseteq SJS=J\). The proof of (1) is the same without the zero case.
\end{proof}
