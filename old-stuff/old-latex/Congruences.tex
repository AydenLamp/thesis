
\begin{definition}[Semigroup congruence]
\label{def:congruence}
An equivalence relation \(\sim\) on a semigroup \(S\) is a \emph{congruence} if it is stable under multiplication: \(s\sim t\) implies \(x s y \sim x t y\) for all \(x,y\in S^1\).
\uses{rem:S1}
\end{definition}

\begin{lemma}[Quotient by a congruence]
\label{lem:quotient-by-congruence}
If \(\sim\) is a congruence on \(S\), the set \(S/{\sim}\) of equivalence classes is a semigroup under \([s]\cdot [t]:=[st]\). The canonical projection \(\pi:S\to S/{\sim}\) is a surjective semigroup morphism.
\uses{def:congruence,def:semigroup-quotient}
\end{lemma}
\begin{proof}
Well-definedness and associativity follow from stability and associativity in \(S\). Surjectivity and multiplicativity of \(\pi\) are immediate.
\end{proof}

\begin{definition}[Rees congruence]
\label{def:rees-congruence}
If \(I\) is an ideal of \(S\), the \emph{Rees congruence} \(\equiv_I\) identifies all elements of \(I\) and keeps distinct elements of \(S\setminus I\): \(s\equiv_I t\iff (s=t)\ \text{or}\ (s,t\in I)\). The quotient \(S/I:=S/{\equiv_I}\) has support \((S\setminus I)\cup\{0\}\) with multiplication
\[
s\ast t=\begin{cases}
st,& s,t,st\notin I,\\[2pt]
0,& \text{otherwise.}
\end{cases}
\]
\uses{def:ideal,def:congruence,def:semigroup-quotient}
\end{definition}

\begin{definition}[Syntactic congruence]
\label{def:syntactic-congruence}
Given \(P\subseteq S\), the \emph{syntactic congruence} \(\sim_P\) is
\[
s\sim_P t\quad:\Longleftrightarrow\quad \forall x,y\in S^1,\ \ x s y\in P\ \Leftrightarrow\ x t y\in P.
\]
The quotient \(S/{\sim_P}\) is the \emph{syntactic semigroup} of \(P\) in \(S\).
\uses{def:congruence,def:semigroup-quotient,rem:S1}
\end{definition}

\begin{definition}[Congruence generated by a relation]
\label{def:generated-congruence}
For a (symmetric) relation \(R\subseteq S\times S\), the \emph{congruence generated by \(R\)} is the intersection of all congruences containing \(R\).
\uses{def:congruence}
\end{definition}

\begin{lemma}[Description of generated congruence]
\label{lem:generated-congruence-description}
If \(R\subseteq S\times S\) is symmetric, the congruence it generates is the reflexive-transitive closure of
\[
\overline{R}\ :=\ \{(x r y,\ x s y)\mid (r,s)\in R,\ x,y\in S^1\}.
\]
\uses{def:generated-congruence,rem:S1}
\end{lemma}
\begin{proof}
Any congruence containing \(R\) contains \(\overline{R}\) and hence its reflexive-transitive closure \(\overline{R}^{\,*}\). Conversely, \(\overline{R}^{\,*}\) is readily checked to be a congruence: if \(u\to v\) is a step from \(\overline{R}\), then \(x u y\to x v y\) is again a step for any \(x,y\in S^1\); closures preserve this property.
\end{proof}

\begin{definition}[Nuclear congruence]
\label{def:nuclear-congruence}
For a morphism \(\varphi:S\to T\), the \emph{nuclear congruence} \(\sim_\varphi\) on \(S\) is defined by \(x\sim_\varphi y \iff \varphi(x)=\varphi(y)\).
\uses{def:semigroup-morphism,def:congruence}
\end{definition}

\begin{theorem}[First isomorphism theorem]
\label{thm:first-isomorphism}
Let \(\varphi:S\to T\) be a semigroup morphism and \(\pi:S\to S/{\sim_\varphi}\) the quotient map. There exists a unique semigroup morphism \(\widetilde{\varphi}:S/{\sim_\varphi}\to T\) with \(\varphi=\widetilde{\varphi}\circ \pi\). Moreover, \(\widetilde{\varphi}\) is an isomorphism \(S/{\sim_\varphi}\cong \varphi(S)\).
\uses{def:nuclear-congruence,lem:quotient-by-congruence,def:isomorphism}
\end{theorem}
\begin{proof}
Define \(\widetilde{\varphi}([x])=\varphi(x)\); this is well-defined by definition of \(\sim_\varphi\), multiplicative, and has image \(\varphi(S)\). It is bijective onto \(\varphi(S)\) with inverse given by \([x]\leftarrow \varphi(x)\).
\end{proof}

\begin{theorem}[Second isomorphism theorem for congruences]
\label{thm:second-isomorphism}
Let \(\sim_1,\sim_2\) be congruences on \(S\) with \(\sim_2\) coarser than \(\sim_1\). Then there is a unique surjective morphism \(\Pi:S/{\sim_1}\to S/{\sim_2}\) such that \(\Pi\circ \pi_1=\pi_2\), where \(\pi_i:S\to S/{\sim_i}\) are the projections.
\uses{def:congruence,lem:quotient-by-congruence}
\end{theorem}
\begin{proof}
Define \(\Pi([s]_1):=[s]_2\); this is well-defined since \(\sim_1\subseteq \sim_2\), multiplicative, and clearly surjective.
\end{proof}

\begin{lemma}[Intersection embeds in a product]
\label{lem:intersection-embeds-into-product}
Let \((\sim_i)_{i\in I}\) be congruences on \(S\) and \(\sim=\bigcap_i \sim_i\). Then \(S/{\sim}\) embeds into \(\prod_{i\in I} S/{\sim_i}\) as a subsemigroup.
\uses{def:congruence,lem:quotient-by-congruence,def:semigroup-product,def:isomorphism,def:subsemigroup}
\end{lemma}
\begin{proof}
Consider \(\pi=(\pi_i)_{i\in I}:S\to \prod_i S/{\sim_i}\). Its nuclear congruence is \(\sim\). By Theorem~\ref{thm:first-isomorphism}, \(S/{\sim}\cong \pi(S)\), a subsemigroup of the product.
\end{proof}

\begin{lemma}[Two distinct \(0\)-minimal ideals give a product embedding]
\label{lem:zero-minimal-ideals-product-embed}
If \(S\) has (at least) two distinct \(0\)-minimal ideals \(I_1,I_2\), then \(S\) embeds into \(S/I_1 \times S/I_2\) as a subsemigroup.
\uses{def:minimal-ideal,def:rees-congruence,lem:intersection-embeds-into-product}
\end{lemma}
\begin{proof}
Since \(I_1\cap I_2=\{0\}\), the intersection of the Rees congruences \(\equiv_{I_1}\) and \(\equiv_{I_2}\) is equality. Apply Lemma~\ref{lem:intersection-embeds-into-product}.
\end{proof}

\begin{definition}[Congruences of ordered monoids]
\label{def:ordered-monoid-congruence}
Let \((M,\le)\) be an ordered monoid. A \emph{congruence of ordered monoids} is a stable preorder \(\preccurlyeq\) on \(M\) that is coarser than \(\le\) and is compatible with multiplication: \(x\preccurlyeq y\Rightarrow a x b \preccurlyeq a y b\). Writing \(x\sim y\) for the induced equivalence \(x\preccurlyeq y\ \&\ y\preccurlyeq x\), the quotient \(M/{\sim}\) carries a well-defined ordered-monoid structure with the order induced by \(\preccurlyeq\), and the projection \(M\to (M/{\sim},\le)\) is a morphism of ordered monoids.
\uses{def:ordered-structures,def:ordered-monoid-morphism,lem:quotient-by-congruence}
\end{definition}