\section{R--L commutativity}

\begin{lemma}[Commutativity of \(\le_L\) and \(\le_R\)]
\label{lem:rLE_lLE_comm}
For any \(x,y \in M\), there exists an element \(z\) with \(x \le_L z\) and \(z \le_R y\) if and only if there exists an element \(w\) with \(x \le_R w\) and \(w \le_L y\).
\lean{rLE_lLE_comm}
\leanok
\uses{def:RRel, def:LRel, lem:RRel-mul_left_compat, lem:LRel-mul_right_compat}
\end{lemma}
\begin{proof}
\leanok
First suppose there exists \(z\in M\) such that \(x \le_L z\) and \(z \le_R y\).  Unwinding the definitions, there are witnesses \(u\) and \(v\) with \(u\cdot z = x\) and \(y\cdot v = z\).  Taking \(w = u\cdot y\) we have
\[
  w \le_L y \quad\text{since}\quad w = u\cdot y
\]
and
\[
  x = u\cdot z = u\cdot (y\cdot v) = (u\cdot y)\cdot v = w\cdot v,
\]
so \(x \le_R w\).  Conversely, suppose there exists \(w\in M\) with \(x \le_R w\) and \(w \le_L y\).  Then there are witnesses \(v\) and \(u\) with \(w\cdot v = x\) and \(u\cdot y = w\).  Let \(z = y\cdot v\).  We compute
\[
  z \le_R y \quad\text{because}\quad y\cdot v = z,
\]
and
\[
  x = w\cdot v = (u\cdot y)\cdot v = u\cdot (y\cdot v) = u\cdot z,
\]
showing \(x \le_L z\).  This establishes the equivalence.
\uses{def:RRel, def:LRel}
\end{proof}

\begin{lemma}[Commutativity of right and left equivalence]
\label{lem:rEquiv_lEquiv_comm}
For any \(x,y \in M\), there exists \(z\in M\) such that \(x\) is left equivalent to \(z\) and \(z\) is right equivalent to \(y\) if and only if there exists \(w\in M\) such that \(x\) is right equivalent to \(w\) and \(w\) is left equivalent to \(y\).
\lean{rEquiv_lEquiv_comm}
\leanok
\uses{def:REquiv, def:LEquiv, lem:rLE_lLE_comm, lem:REquiv-mul_left_compat, lem:LEquiv-mul_right_compat}
\end{lemma}
\begin{proof}
\leanok
Suppose there exists \(z\) with \(x\) left equivalent to \(z\) and \(z\) right equivalent to \(y\).  Write this as \(x \le_L z \land z \le_L x\) and \(z \le_R y \land y \le_R z\).  From the left and right preorder conditions we have witnesses \(u_1,u_2,v_1,v_2\) satisfying
\[
  u_1\cdot z = x,\qquad z\cdot u_2 = x,\qquad y\cdot v_1 = z,\qquad z\cdot v_2 = y.
\]
Set \(w = u_1\cdot y\).  Then using the first pair of equations we have
\[
  w \le_L y \quad\text{since}\quad w = u_1\cdot y,
\]
and from the remaining equations we compute
\[
  x = u_1\cdot z = u_1\cdot (y\cdot v_1) = (u_1\cdot y)\cdot v_1 = w\cdot v_1,
\]
so \(x \le_R w\).  Moreover, by applying Lemma~\ref{lem:REquiv-mul_left_compat} to the right equivalence \(z \sim_R y\) we deduce \(u_1\cdot z\) is right equivalent to \(u_1\cdot y\).  Since \(u_1\cdot z = x\), this shows \(x \sim_R w\).  Combined with the previous observation that \(w\) and \(y\) are left equivalent, we obtain the right-to-left implication.

Conversely, assume there exists \(w\) such that \(x\sim_R w\) and \(w\sim_L y\).  This means \(x \le_R w\) and \(w \le_R x\), and \(w \le_L y\) and \(y \le_L w\).  Pick witnesses \(v_1,v_2,u_1,u_2\) with
\[
  w\cdot v_1 = x,\qquad x\cdot v_2 = w,\qquad u_1\cdot y = w,\qquad y\cdot u_2 = w.
\]
Let \(z = y\cdot v_1\).  Then \(z\) and \(y\) are right equivalent because both \(y \le_R z\) and \(z \le_R y\) hold via the witnesses above.  Using Lemma~\ref{lem:LEquiv-mul_right_compat} applied to the left equivalence \(w \sim_L y\), we have \(x\cdot u_1\) is left equivalent to \(w\cdot u_1 = u_1\cdot y\).  Since \(x\cdot u_1 = (w\cdot v_1)\cdot u_1\) and \(z = y\cdot v_1\), a straightforward computation shows \(x \le_L z\) and \(z \le_L x\).  Hence \(x\sim_L z\) and \(z\sim_R y\), completing the proof.
\uses{def:REquiv, def:LEquiv}
\end{proof}
